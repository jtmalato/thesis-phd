ABSTRACT
- comparison of ME/CFS and MS diseases
- dataset related to IgG Ab responses to 6 herpesviruses
- analysis of binary symptoms

OBJECTIVE 1
- report the symptomatoly and its association with herpesvirus IgG Abs using data from 4 disease-trigger-based subgroups of ME/CFS and MS patients
OBJECTIVE 2
- assess whether serological data could distinguish ME/CFS and subgroup from MS using SL algorithm
OVERALL
- assess new ways to difefrentiate ME/CFS and its subgroups from MS
- the findings for MS are not particularly novel...


Symptoms:
    RESULTS ME/CFS
    - positive association: HSV1 Ab concentrations and brain fog
    RESULTS MS
    - negative association: CMV Ab concentrations and eyesight disturbances
    - positive association: EBV Ab concentrations and bladder problems
    - stronger symptom-herpesviruses associations than in ME/CFS
Models:
    - distinguish 3 ME/CFS subgroups from MS

CONCLUSION
- IgG Ab data explain better the symptomatology of MS than ME/CFS

- STILL to be done: longitudinal study to assess the natural course of the disease



% ----------------------------------------------------------------------
% INTRO

- main candidate proteins for the deleterious autoimmune phenomenon are the adrenegic receptors \citep{bynke2020AutoantibodiesBetaadrenergic, freitag2021AutoantibodiesVasoregulative, loebel2016AntibodiesAdrenergic}.
- also, other human proteins such as Anoctamin-2 and thyroid peroxidase, have also been suggested \citep{loebel2017SerologicalProfiling, sepulveda2021ImpactGenetic}.

ME/CFS vs MS \citep{loebel2017SerologicalProfiling, ramosRegulatoryNaturalKiller2016}

% ----------------------------------------------------------------------
% MATERIALS and METHODS

- 222 ME/CFS patients
    - 42 S0
    - 42 S1
    - 92 S2
    - 46 S3
- 46 MS patients

Table 1: Partient description

% ----------------------------------------------------------------------
% SYMPTOMS

- Non infectious ME/CFS groups had more similar symptom profile with MS than the other infected groups
    - MS might be composed of mild/moderate symptomatology
    - MS should also repor severities!

Figure 1: Age-adjusted OR for the presence of each of 48 symptoms when comparing the whole ME/CFS group to the MS group

Figure 2: Age-adjusted OR for the presence of each of 48 symptoms when comparing subgroups of ME/CFS to the MS group

% ----------------------------------------------------------------------
% UNIVARIATE ANALYSIS IGG AB DATA

Table 2: Seroprevalence, median concentration and respective IQR per study group and herpesvirus IgG antibody

Supp table 3:
- Model to discriminate MS patients from healthy controls
    - increased EBV-VCA
    - reduced HSV1

- Combining data from multiple Abs could help discriminating the two groups of patients (S1 vs MS)

% ----------------------------------------------------------------------
% DISCUSSION

- symptoms from the immunological domain are essential to diagnose ME/CFS and differentiate MS

- symptoms might be subjective to each individual
- MS have higher number of significant Ab-symptom associations
    - herpesviruses have higher impact on this group than on ME/CFS

Associations:
MS
    - alternate results: proposal that the reduction of Abs against CMV might result in low-grade ocular infection in MS patients
    - EBV-EBNA1 and bladder problems; also a common dysfunction in MS
ME/CFS
    - VZV and bladder problems; also a common dysfunction in S3
    - ME/CFS and increase branc fog with HSV1 (positive assocaition) -- HSV1 is known to be neurotropic during latency \citep{marcocci2020HerpesSimplex}
        - this link between HSV1 IgG antibodies and cognitive deficits has been shown in
                - HC \citep{fruchter2015ImpactHerpes, jonker2014AssociationExposure, tarter2014PersistentViral}
                - Alzheimer's disease \citep{murphy2021HerpesSimplex}
                - schizophrenia \citep{dickerson2020AssociationExposure}
                - bipolar disorder \citep{tucker2019AssessmentCognitive}

- Stratification of patients can reduce the overall power and reduce certain findings (this related the the reduction in significant associations post multiple testing adjustment)

- We were able to discriminate 3 subgroups from MS
    - due to increased IgG Ab concentrations in MS rather than in ME/CFS (corroborated by \citet{loebel2017SerologicalProfiling} with regarts to EBV antigens)

- even considering the negative results when studying candidate biomarkers for ME/CFS, all goes to a process of incremental steps and collaborative efforts towards an effective diagnosis, treatment, or cure \citep{lacerda2019HopeDisappointment}.


% Main limitations to this study
1. single snapshot of the patients
2. IgG Abs are used as biomarkers of a past active infection -- limited power to judge when the last active infection occurred and whether the hrpvs are currently active or simply latent
    - use IgM Abs in the same biological samples
3. Convenience data (from MS) = "pure patients" or "mild cases"
4. disease course profiles do not comply with standard classification used in MS research (reviewed in \citet{klineova2018ClinicalCourse}) -- i.e., lack of interpretation of the disease stage


% TO DOS
- Combining data from multiple Abs could help discriminating the two groups of patients (S1 vs MS)
- (regarding positive and negative associations between herpesvirus Ab concentration and different symptoms in each group) Conduct a longitudinal study with multiple timepoints and check whether the association remain valid during follow-up
- verify this analysis but with other ME/CFS-stratification settings
- same experience but this time looking at IgM antibodies (re-analysis) to validate possible (time-related) relations to viral infection thatn were not able to be interpreted with IgG

- longitudinal study with at least 3 timepoints -- help discenr the robustness of the reported associations during disease course (and also study the disease progression) and whether they could be targeted by any existing drug treatment