\section{Introduction}

- misdiagnosis (relative)
    - lack of a consensual case definition \citep{malato2021Statisticalchallenges}
    - consensual case definition are both difficult find and suboptimal to patien/control discrimination \citep{jason2014ExaminingCase}
    - more related to a discussion about the different case definitions implemented \citep{brurberg2014CaseDefinitions, lim2020ReviewCase}
- misdiagnosis (in a strict sense) can result from
    - random fluctuations in the natural, pathological process of the exclusionary disease (e.g., low-graded or relapsing MS)
    - limited resources to run the battery of test necessary to exclude all known diseases that could explain fatigue
    - ambiguity around the exclusionary criteria themselves \citep{jason2023EstablishingConsensus}
    - given the state of research, this type of misdiagnosis seems inevitably present in ME/CFS studies \citep{nacul2019HowHave}
    
- study simulation to
    - determine statistical power of detecting associations with ME/CFS under misdiagnosis in a strict sense
    - investigate the impact of imperfect sensitivity/specificity for the presence of a given antibody that could be causing ME/CFS

\section{Statistical methodology}

Definitions
    - candidate causal factor
        - genetic factor
        - occurrence of a given infection


\subsection{Formulation of the problem}

- Table 1: Two-way contingency table of a typical case-control study

- We worked assuming an binary output of presence/absent or infected/non-infected

- This work could be expanded to consider
    - different candidate causal factor probabilities within the ME/CFS patients (assuming that this cohort is composed not only by possible misdiagnosed individuals, but also by different sub-types of the disease)

Assumptions:
    1. ME/CFS-diagnosed cases are a mix of apparent and genuine patients of the disease
    2. The causal factor is only associated with genuine ME/CFS patients
    3. Apparent cases are similar to healthy controls as far as the association with the causal factor is concerned
    4. The chance of an ME/CFS misdiagnosis is only dependent on the true clinical status of the cases and not on the confounding factors
    5. The true association is independent of disease duration and disease triggers, among other factors occurring during the disease course
    6. Healthy controls were not misdiagnosed as such
    7. The value of the candidate causal factor can be determined perfectly in each individual

- impact of sensitivity/specificity misdiagnosis: the candidate causal factor cannot be determined perfectly in each individual
    - serological studies that aim to investigate whether the presence of specific antibodies is associated to ME/CFS \citep{ruiz-pablos2021EpsteinBarrVirus} or whether these antibodies can be used for disease diagnosis \citep{sepulveda2022RevisitingIgG}

Assumptions:
    7. There are only two possible serological outcomes for each individual: seronegative or seropositive
    8. The sensitivity and specificity of the serological classification are identical for all of the individuals


\subsection{Simulation study}

- theta0
    - genetic association: minor allele frequency of a given SNP in the healthy population


\subsection{Application to two ME/CFS studies}


\section{Results}

\subsection{Simulation study: impact of ME/CFS misdiagnosis}

Figure 1: Probabilities of detecting an association as a function of the misdiagnosis rate

Table 2: Maximum values of misdiagnosis probability that ensure the minimum power of 80\% to detect a genuine association as a function of theta0 and sample size per group
    - delta = 10; gamma = 0.53 (n >= 100), irrespective of theta0
    - delta = 5; gamma = 0.24 (n >= 100), irrespective of theta0
    - delta > 1.25; gamma < 0.88 (n = 2500, 5000), irrespective of theta0
    
    - delta = 1.25, 1.5; gamma = BAD (n = 100), irrespective of theta0
    - delta = 1.5; gamma = 0.8 (n = 2500, 5000), irrespective of theta0
    - for a weak association, the chance of finding reproducible results was very low, even under the assumption of a perfect diagnosis!
    - As consequence, testing "common disease, common variant hypothesis" in ME/CFS is likely to fail in future genetic associations


\subsection{Simulation study: impact of ME/CFS misdiagnosis and misclassification on the candidate causal factor}

Figure 2: Probabilities of detecting an association as a function of the misdiagnosis rate considering different combinations of sensitivity and specificity

Table 3: Maximum values of misdiagnosis probability that still ensures a power of rejecting the null hypotheses of at least 80\% for $\Delta_T = 3$ and $\theta_0 = 0.25$
    - The additional assumption of imperfect classification of the candidate causal factor would make the previously estimated power even worse!
    - The results for Figure 2 and Table 3 don't even replicate the reality of ME/CFS and a more optimistic scenario with OR=3 had to be simulated instead!


\subsection{Application to data from two ME/CFS studies}

Table 4: Reported associations of a candidate gene association study \citep{steiner2020AutoimmunityRelatedRisk}
    - We could only assume replication of one finding (rs3087243 in \textit{CTLA4}) under misdiagnosis of less than 9\% of the population, otherwise the power decreases below the idealised threshold of 80\%.
    - For rs2476601 in \textit{PTPN22}, whose association was reported significant in the original study, a misdiagnosis of 10\% of the ME/CFS population returns a power around 50\%, which is essentially the same as flipping a coin for the association.

Table 5: Summary of serological findings from \citet{cliff2019CellularImmune}


Figure 3: The relationship between the misdiagnosis probability and the probability of detecting an association estimated from simulated data from two previously published studies

\section{Discussion}

- strong associations with ME/CFS can be detected with reasonable power under a non-negligible misdiagnosis rate
- however strong associations might not be the case of ME/CFS, giving the difficulty in finding biomarkers \citep{scheibenbogen2017EuropeanME} and a clear genetic signature of the disease \citep{dibble2020GeneticRisk, hajdarevic2022GeneticAssociation, herrera2018GenomeepigenomeInteractions, tanigawa2019ComponentsGenetic}
- sample sizes larger than 500 individuals per study group are able to compensate for the reduction in power due to misdiagnosis alone
- larger studies are becoming more common in well-known and highly-funded diseases, such as cancer, cardiovascular diseases \citep{giri2019TransethnicAssociation}, and autoimmune disorders \citep{bjornevik2022LongitudinalAnalysis, internationalmultiplesclerosisgeneticsconsortiumimsgc2013AnalysisImmunerelated}
- large studies in ME/CFS are being proposed, such as DecodeME [link], but in all, large ME/CFS studies are currently unfeasible due to limited funding and poor societal recognition of the disease \citep{pheby2021LiteratureReview}
- use of data from biobanks with larger number of biological samples \citep{lacerda2018UKME}
- conduct multi-centric studies \citep{scheibenbogen2017EuropeanME}

- Healthy controls and ME/CFS can show similar levels of fatigue \citep{cella2010MeasuringFatigue, malato2021Statisticalchallenges}
- Again, self-reported healthy controls
    - a thorough clinical assessment should also be performed in putative healthy controls

From a purely mathematical viewpoint, the proposed idea of patient misdiagnosis (in a strict sense) is equivalent to defining two subgroups within the population of ME/CFS suspected cases. This makes our results directly applicable to an alternative study on stratification of patients.
- genomic data suggests taht ME/CFS might not be one but several diseases under the same umbrella term \citep{kerr2008GeneExpression, zhang2010MicrobialInfections}
- Extend our study and interpred the different possible degrees of association as different subgroups of ME/CFS patients

- sample size of 500--1000 individuals per study is a minimal requirement to detect mild-to-moderate associations with a high power under the assumption of misdiagnosis (multi-centric studies which require collaboration amongst ME/CFS researchers).
- under the impossibility to increase cohort sample size, the focus should be on reducing the rate of strict misdiagnosis, i.e., following existing recommendations for research reports on ME/CFS ( reporting the screening laboratory tests and the cut-off values for exclusion \citep{jason2012MinimumData}), and by continued search of alternative diagnoses and co-morbidities \citep{nacul2021EuropeanNetwork}.
- Study more homogeneous cohorts of patients, where a chance of misdiagnosis is reduced!


\section{CONCLUSIONS}

Dfinitions of
    - misdiagnosis: different case definitions are used by clinicians (relative misdiagnosis)
    - misclassification: failing the genuine diagnosis of another disease (misdiagnosis in a strict sense)

- estimated the power needed to detect a genuine association between a potential causal factor and ME/CFS
- Current ME/CFS studies have suboptimal power under the assumption of misdiagnosis
- This power can be improved by increasing the overall sample size using multi-centric studies, reporting the excluded illnesses and their exclusion criteria, or focusing on a homogeneous cohort of ME/CFS patients with a specific pathological mechanism where the chance of misdiagnosis is reduced

- We worked assuming an binary output of presence/absent or infected/non-infected


- This work could be expanded to consider
    - different candidate causal factor probabilities within the ME/CFS patients (assuming that this cohort is composed not only by possible misdiagnosed individuals, but also by different sub-types of the disease)

- impact of sensitivity/specificity misdiagnosis: the candidate causal factor cannot be determined perfectly in each individual
    - serological studies that aim to investigate whether the presence of specific antibodies is associated to ME/CFS \citep{ruiz-pablos2021EpsteinBarrVirus} or whether these antibodies can be used for disease diagnosis \citep{sepulveda2022RevisitingIgG}

