\chapter{Additional information to Chapter~\ref{chapter:impact-misdignosis-2023}}
\label{appendix:impact-misdignosis-2023}

%%%%%%%%%%%%%%%%%%%%%%%%%%%%%%%%%%%%%%%%%%%%%%%%%%%%%%%%%%%%%%%%%%%%%%%%
%%%%%%%%%%%%%%%%%%%%%%%%%%%%%%%%%%%%%%%%%%%%%%%%%%%%%%%%%%%%%%%%%%%%%%%%
\section{Supplementary tables}


\begin{table}[h]
    \centering
    \caption[Augmented version of the observed $2 \times 2$ contingency table in the presence of the misdiagnosis of ME/CFS cases for a classical case-control association study]{Augmented version of the observed $2 \times 2$ contingency table in the presence of the misdiagnosis of ME/CFS cases for a classical case-control association study. Parameter $\theta_0$ is the probability of the presence of the candidate causal factor shared across healthy controls and apparent (false positive) ME/CFS cases, $\theta_1^*$ is the true probability of the causal factor in the true ME/CFS patients. Misdiagnosis probability is given by the parameter $\gamma$.}
    \begin{tabular}{lcccccc} 
\toprule
\multirow{2}{*}{\begin{tabular}[c]{@{}l@{}}\\Causal factor\end{tabular}} &  & \multirow{2}{*}{Controls} &  & \multicolumn{3}{c}{ME/CFS-diagnosed cases}                   \\ 
\cmidrule{5-7}
                                                                           &  &                           &  & (Apparent)           &  & (True)                      \\ 
\midrule
Present                                                                       &  & $\theta_0$                &  & $\gamma\theta_0$     &  & $(1-\gamma)\theta_1^*$      \\
Absent                                                                  &  & $1-\theta_0$              &  & $\gamma(1-\theta_0)$ &  & $(1-\gamma)(1-\theta_1^*)$  \\
\bottomrule
\end{tabular}
    \label{appendix:candidate-gene-probabilities}
\end{table}
% Supplementary Table 1
% Supplementary Table~\ref{appendix:candidate-gene-probabilities}


\begin{table}[h]
    \centering
    \caption[Augmented version of the observable $2 \times 2$ contingency table in the case-control association study with possible misdiagnosis of ME/CFS cases and misclassification of the true serological status]{Augmented version of the observable $2 \times 2$ contingency table in the case-control association study with possible misdiagnosis of ME/CFS cases and misclassification of the true serological status (seropositive, $S^+$, and seronegative, $S^-$). Parameter $\theta_0$ is the probability of the presence of the candidate causal factor in healthy controls and apparent (false positive) ME/CFS cases, $\theta_1^*$ is the true probability of the causal factor in the true ME/CFS patients. Misdiagnosis probability is modulated by the parameter $\gamma$. The true serological status is dependent on the sensitivity ($\pi_{se}$) and specificity ($\pi_{sp}$) of the serological test.}
    \resizebox{\textwidth}{!}{\begin{tabular}{ccccc} 
\toprule
\multirow{2}{*}{\begin{tabular}[c]{@{}c@{}}Estimated \\Serological status\end{tabular}} & \multirow{2}{*}{\begin{tabular}[c]{@{}c@{}}True  \\serological status\end{tabular}} & \multirow{2}{*}{Controls}  & \multicolumn{2}{c}{ME/CFS-diagnosed cases} \\ 
\cline{4-5}
                                                                                 &                &                            & (Apparent)                       & (True)                                \\ 
\hline
\multirow{2}{*}{$S^+$}                                                    & $S^+$            & $\pi_{se}\theta_0$         & $\pi_{se}\gamma\theta_0$         & $\pi_{se}(1-\gamma)\theta_1^*$        \\
                                                                                 & $S^-$ & $(1-\pi_{sp})(1-\theta_0)$ & $(1-\pi_{sp})\gamma(1-\theta_0)$ & $(1-\pi_{sp})(1-\gamma)(1-\theta_1^*)$ \\
\cmidrule{2-5}
\multirow{2}{*}{$S^-$}                                                    & $S^+$            & $(1-\pi_{se})\theta_0$     & $(1-\pi_{se})\gamma\theta_0$     & $(1-\pi_{se})(1-\gamma)\theta_1^*$    \\
                                                                                 & $S^-$ & $\pi_{sp}(1-\theta_0)$     & $\pi_{sp}\gamma(1-\theta_0)$     & $\pi_{sp}(1-\gamma)(1-\theta_1^*)$    \\
\bottomrule
\end{tabular}}
    \label{appendix:serology-probabilities}
\end{table}
% Supplementary Table 2
% Supplementary Table~\ref{appendix:serology-probabilities}


%%%%%%%%%%%%%%%%%%%%%%%%%%%%%%%%%%%%%%%%%%%%%%%%%%%%%%%%%%%%%%%%%%%%%%%%
%%%%%%%%%%%%%%%%%%%%%%%%%%%%%%%%%%%%%%%%%%%%%%%%%%%%%%%%%%%%%%%%%%%%%%%%
\section{Supplementary equations}

We constructed our analysis considering a classical epidemiological scenario where for a single putative risk factor, individuals can be divided into exposed versus non-exposed. This result can be summarised by a $2 \times 2$ contingency table, whose sampling distribution is the product of two independent Binomial distributions, one Binomial distribution per group,
% 
\begin{equation}
    f(x_i | n_i; \theta_i) = \prod_{i = 0,1} \binom{n_i}{x_i} \theta_{i}^{x_i} (1-\theta_i)^{n_i - x_i}\ ,
    \label{appendix:sample-distribution-01}
\end{equation}
% Supplementary Equation (1)
% Supplementary Equation~(\ref{appendix:sample-distribution-01})
% 
where $x_0$ and $x_1$ are the observed frequencies of healthy controls and suspected cases with presence of the candidate causal factor, respectively, $n_0$ and $n_1$ are the corresponding sample sizes of each group, and $\theta_0$ and $\theta_1$ are the probabilities for the presence of the candidate causal factor in healthy controls and suspected cases, respectively.

%%%%%%%%%%%%%%%%%%%%%%%%%%%%%%%%%%%%%%%%%%%%%%%%%%%%%%%%%%%%%%%%%%%%%%%%

\begin{equation}
    \theta_{1}^{*} = \frac{\theta_0 \Delta_T}{1 + \theta_0 \left(\Delta_T - 1\right)}
    \label{appendix:true-t1}
\end{equation}
% Supplementary Equation (2)
% Supplementary Equation~(\ref{appendix:true-t1})