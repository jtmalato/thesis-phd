\section{Introduction}

- prevalence: 0.4--1.0\%
- female-male ratio: 6:1
- >20 different case definitions based on symptoms
- possible false positive patients can affect research on ME/CFS
- UKMEB


\section{UKMEB}

- Biobank description
- 523 Participants
    - 275 ME/CFS
    - 136 HC
    - 112 MS


\section{Diagnostic agreement analysis}

- Description of 4 most commonly used ME/CFS case definitions:
    - CDC-1994
    - CCC-2003
    - ICC-2011
    - IOM-2015

- Table 1: Frequency of suspected cases of ME/CFS according to their diagnostic outcomes using different case definitions
    - exploratory analysis on agreement

- Table 2: Estimates of the Jaccard's similarity index for the four case definitions of ME/CFS
    - Jaccard similarity index
    - J range = 0.752--0.876


\section{Symptoms’ similarity analysis}

- Figure 1: Symptom's similarity analysis based on the Cohen's kappa coefficient
    - A: MDS
    - B: hierarchical cluster analysis


\section{Impact of misclassification on an association analysis}

Assumptions:
    1. suspected cases could be divided into apparent (or false positive) cases and true positive cases of ME/CFS
    2. the apparent cases were deemed equivalent to healthy controls in terms of degree of exposure, i.e., the probability of exposure in these individuals was given by $\theta_0$
    3. there was an overall misclassification rate, $\gamma$, for the suspected cases
    4. misclassification was only dependent on the true clinical status of each suspected case. 

Table 3: Augmented version of the observable 2x2 frequency table and probabilities for healthy controls and suspected cases

Figure 2: Estimated probability of rejecting H0 as function of the misclassification rate


\section{Concluding remarks}


\section{CONCLUSIONS}

1. Diagnostic agreement
    - Jaccard index: even if the general practitioners applied two different case definitions of ME/CFS in their diagnosis, there could still be a fraction of suspected cases where the respective diagnostic outcomes might not agree with each other.

2. Symptoms' similarity
    - Patient symptoms are very heterogeneous
    - Patient symptoms overlap with other similar diseases, and even in some extreme cases, healthy controls (Fig 1B)
    - Some ME/CFS diagnosed patients could, in fact, be true cases of another disease \citep{nacul2019HowHave}
    - Results demonstrated the difficulty of diagnosing ME/CFS based on symptoms' assessment alone
        - To overcome this and other difficulties there are currently efforts for a stronger collaboration among European researchers for accelerating the discovery of an objective disease-specific biomarker \citep{scheibenbogen2017EuropeanME}

3. Simulation study
    - ME/CFS can be divided into different groups of true cases (true positives) and apparent cases (false positives)
    - 

- Before going into biomarker discovery: biomarker discovery is very likely to suffer from limited statistical power due to a possible misclassification of the suspected cases
    - a possible solution is to take misclassification into account in the respective analysis
    - this approximation towards reality could in turn become a problem of overparameterization, which could be avoided by
        1. frequentist way: fixing misclassification rate in a reasonable estimate for the sensitivity of the diagnosis test
        2. Bayesian way: prior information about misclassification rate takes the form of a probability distribution
        - both approaches show a limitation: lack of biomarker for ME/CFS = unclear what's the reasonable value for the probability distribution to choose for the sensitivity of current diagnostic tools
    