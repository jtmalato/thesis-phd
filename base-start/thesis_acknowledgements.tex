\section*{Acknowledgments}

\addcontentsline{toc}{section}{Acknowledgments}

%%%%%%%%%%%%%%%%%%%%%%%%%%%%%%%%%%%%%%%%%%%%%%%%%%%%%%%%%%%%%%%%%%%%%%%%
% A few words about the university, financial support, research advisor, dissertation readers, faculty or other professors, lab mates, other friends and family...

% The completion of this work stands as a testament to resilience and dedication in navigating unforeseen circumstances.
This thesis was financially supported by Fundação para a Ciência e a Tecnologia (grant reference SFRH/BD/149758/2019) and was hosted by the Instituto de Medicina Molecular João Lobo Antunes at the Faculty of Medicine, University of Lisbon.

\bsni
% ----------------------------------------------------------------------
This work was produced despite setbacks and delays in data transfer agreements due to Brexit, isolation and salary payment delays due to the \covid pandemic, and having to shorten academic visits abroad due to the ongoing Russo-Ukrainian War.
Over the course of the last four years I inhabited four different homes, suffered from hearing loss, and even did daily therapy on a hyperbaric chamber for a while.
During this time, I also experienced the joys of marriage, welcomed my first nephew, and made numerous friends who have enriched my life.
% and to whom I want to thank.

% ----------------------------------------------------------------------
As such, I would like to express my gratitude to all the people who directly and indirectly contributed to and provided invaluable support for this thesis.
% I have been fortunate enough to develop this work with the support of .
% As such, I would like to thank:

% \bsni
% I was fortunate enough to...
% ----------------------------------------------------------------------
Nuno Sepúlveda, my supervisor and mentor.
You showed me the magnificent world that lies at the intersection between life sciences and statistics.
Six years ago, I started working alongside you as an unproven Biology student (with highly doubtful programming skills).
Your expertise, dedication, and genuine curiosity towards science, together with the stubbornness to do things ``just because'' and think actively instead, have become an inspiring and positive reference.
% Your stubbornness to do things ``just because'' and think actively instead, together with 
% Your vocal irreverence to do things ``just because'' and think actively instead, together with 
% , have become an inspiring reference.
% Your irreverence towards things that are done ``just because''
% You taught me to be more rigorous and shift my academic writing away from Saramago's-like prose.
Words cannot express my gratitude to you and everything you have taught me.
% I see now that I made some of your views my own
I can only say that it has been a pleasure to discuss everything, anything, and all the things in between.
Muito, muito obrigado.
%   irreverencia relativamente a não seguir e implementar o que see faz porque "sempre see fez"
% Thank you for everything, even the heated discussions
% I cannot enumerate the countless heated discussion on 
% , and can only hope to continue emulating your views henceforth.

% After six years of working alongside you, I 
% Muito obrigado.

% your mathematical (and frustrating at times) mathematical rigour
% mathematical and at times frustrating rigour

% Over the past six years, your expertise, dedication, and genuine curiosity for research have become an inspiring reference.

% I am extremely grateful for the opportunity to lear from you and that I hope to bring with me.

% You taught me to be more rigorous and shift my academic writing away from Saramago's-like prose.
% You also taught me about the 

% I remember vividly the first time I understood 
% as an unproved and with highly doubtful programming skills
% My main supervisor and mentor, Nuno Sepúlveda.
%   From playing football in London to playing tennis in Warsaw.
%   irreverencia relativamente a não seguir e implementar o que see faz porque "sempre see fez"

% \bsni
% ----------------------------------------------------------------------
% My co-supervisor and self-proclaimed XXX (``desorientador'') Luís Graça.
Luís Graça, my co-supervisor and self-proclaimed ``desorientador'' (Portuguese for disorienting-supervisor).
Thank you for all the insightful discussions and comments during our lab meetings, and for giving me the opportunity to work and analyse official data and develop interesting and impactful work on the topic of \sars (presented in this thesis in Chapter~\ref{chapter:2022-covid19-01} and Chapter~\ref{chapter:2023-covid19-02}).
I am hopeful of what is to come.
% You have created an amazing group of talented young researchers, fostering a welcoming environment whenever I was around.

% ----------------------------------------------------------------------
% Thesis committee? Jorge Carneiro and Ruy Ribeiro

% \bsni
% ----------------------------------------------------------------------
André Fonseca, who started this journey with me.
Your dedication towards work-life balance is remarkable, and your patience and wise comments to my insecurities are only matched by the love I received from your family during my time in Warsaw.
I am sure you will accomplish great things.

% \bsni
% ----------------------------------------------------------------------
The people at the LGraca group.
Rodrigo Pedroso, for cutting straight to the point every single time, and being a great teammate.
Saumya Kumar, who during the first years of my project took upon herself to teach me the ways of knowing my audience, frenetically studying to connect answers from completely different topics, and also Indian cuisine. I still have a .txt file titled ``things-saumya-says'' that I open from time to time.
JC, who keeps making the extra effort to find the holes in my research ideas and then pesters me with counter-arguments.
Miguel Santos, for the endless stream of motivational speeches during the submission of this work. Also, for promptly volunteering to read the introduction of my thesis---without knowing it is almost 40 pages long; and for reading it anyway, in time to give his comments!
To the rest of `The Bunker' people, Tomás Gomes, Diana Matias, Rui Vieira, and Diogo Fonseca, for including me in their den and making me feel welcome coming to work at iMM, during the final and more stressful stages of the thesis.
Filipa Ribeiro, Catarina Mendes, Pedro Ribeiro, Pedro Gaspar, Vladimir Ghilas, Maria Miguel, Deepanwita Ghosh, and all the talented past and present members of the lab, for creating such a fun and stimulating environment to do research at.

% \bsni
% ----------------------------------------------------------------------
My CAML PhD colleagues at iMM, Ana Cachucho, Sara Pinto, and Carol Pacini, for providing enough sangria to spark creativity whenever I felt stuck.
Also, for letting me win at every single virtual board game (suck it Sara!) and ensuring the stability of my mental health during early pandemic isolation.
% You are the best friends someone can have.
% and cheering every minor achievement along the way.

% \bsni
% ----------------------------------------------------------------------
MI$^2$.AI research group and Przemys\l{}aw Biecek, for hosting me at Politechnika Warszawska and providing the fantastic work conditions where I was able to produce most of my written work.
With particular emphasis, thank you to Kasia Wo\'{z}nica, Anna Kozak, and Hubert Baniecki, for the conversations and sporadic hang-outs with lots of tea and cake. You have my contact, come for a visit!

% \bsni
% ----------------------------------------------------------------------
% Luís Nacul, Eliana Lacerda and all at the CureME
Eliana Lacerda and Luís Nacul, representing the CureME research group and the UK ME/CFS Biobank.
Thank you for introducing me to the research done in \cfs and contextualising my data-driven questions with clinical insights.
% , patiently providing me with all the background information.
% always made the effort to provide 
% who were the first to give me 
% introduced me to the the topic of my thesis

% \bsni
% ----------------------------------------------------------------------
Francisco Westermeier, for directly keeping me updated on parallel research in \cfs. And also, for doing it indirectly, by providing Nuno with a continuous stream of ideas to keep tabs on.

\bsni
% ----------------------------------------------------------------------
Agora em Português.
Este trabalho é fruto de muito e continuado apoio.
Assim sendo, quero agradecer a todos aqueles que, directa ou indirectamente, contribuíram para o seu progresso e conclusão.

% ----------------------------------------------------------------------
Marília Antunes, Ruy Ribeiro, e Manuel Carmo Gomes, pelos ensinamentos e disponibilidade em atenderem simpaticamente todas as chamadas e responderem a todos os emails, mesmo quando fora de horas.

% \bsni
% ----------------------------------------------------------------------
Clara Cordeiro, por nunca permitir que incongruências estatísticas passassem por si. E por todo o apoio desde o primeiro dia deste projecto.

% ----------------------------------------------------------------------
Orquestra Académica da Universidade de Lisboa e todos os seus membros.

% ----------------------------------------------------------------------
Grupo de almoços da Cantina Velha, discutir convosco é sempre um prazer.

% ----------------------------------------------------------------------
Constantino Caetano, Miguel Matos, Pedro Moura, Olivia Schroeder, Margarida Camoesas, Rita Bizarro, e Carolina Mendeiro, pela vossa continuada amizade.

% ----------------------------------------------------------------------
Pawe\l{} e Adam, por num encontro aleatório terem conseguido que Varsóvia se tornasse uma segunda casa.

% ----------------------------------------------------------------------
Belinha, Paupério, e respectivas cantoras do rés-do-chão.
Maria Frederica, e Raúl, porque não custa nada adicionar mais uma linha à tese.
Manu, porque sim.

\noindent
% ----------------------------------------------------------------------
Aos membros da minha família:
% \bsni
% ----------------------------------------------------------------------
Mãe e Pité, por incansavelmente zelarem por mim e garantirem que não me disperso em demasia, entre todas as ideias e objectivos que vão surgindo. Devo-vos muito, e um dos grandes prazeres que tenho na vida é poder discutir saúde e ciência convosco.
% ----------------------------------------------------------------------
Pai, por toda a companhia e passeios à distância, e por sempre me puxar para ver o `panorama geral' onde o meu trabalho se insere.
% ----------------------------------------------------------------------
Meus irmãos, Ana, Madalena, e Martim, por toda a companhia, desde sempre; vejo continuamente a vossa influência em tudo o que faço.
% ----------------------------------------------------------------------
Todos os meus tios, tias, primos, e primas, que distribuem tanto carinho quanto desconversa.
% ----------------------------------------------------------------------
Minhas Avós Maria, Karin, e Marília.
% ----------------------------------------------------------------------
Aos Avós que gostava que celebrassem comigo, Catucha, João José, e Nuno.

\bsni
% ----------------------------------------------------------------------
Por fim, e mais que todos, à Dida-Borboleta.
Completas-me e sou feliz contigo todos os dias.


\vfill
