\section*{Abstract}
\addcontentsline{toc}{section}{Abstract}
\mark{{Abstract}}
%%%%%%%%%%%%%%%%%%%%%%%%%%%%%%%%%%%%%%%%%%%%%%%%%%%%%%%%%%%%%%%%%%%%%%%%
% Abstract text, 1200--1500 words minimum. Maximum 5 keywords.
% ----------------------------------------------------------------------
% ----------------------------------------------------------------------
% Intro intro
% This thesis describes research developed on two topics.
% The first topic concerns research strategies in myalgic encephalomyelitis/chronic fatigue syndrome (\cfs) admitting a degree of uncertainty in the population of patients diagnosed with this illness.
% Studies were done acknowledging possible patient misdiagnosis as possible factor hindering the reproducibility of research in this field.
% Stratification of patients into distinct subgroups was also considered, in order to explore potential links between past viral infections and the immune origins of the disease.
% % ----------------------------------------------------------------------
% The second topic is related to the coronavirus disease 2019 (\covid) pandemic and the state of concern and urgency in a public health emergency.
% The risk of \cfs patients towards a severe acute respiratory syndrome coronavirus 2 (\sars) infection was assessed.
% Afterwards, with the rise in the number of cases from the new \sars Omicron BA.5 subvariant, I studied how past infections from other \sars variants and subvariants would reduce the risk of new infections, in the context of highly vaccinated populations.
%%%%%%%%%%%%%%%%%%%%%%%%%%%%%%%%%%%%%%%%%%%%%%%%%%%%%%%%%%%%%%%%%%%%%%%%
% Abstract text, 1200--1500 words minimum. Maximum 5 keywords.
% ----------------------------------------------------------------------
% Intro
This thesis describes research developed in two topics: research strategies in myalgic encephalomyelitis/chronic fatigue syndrome (ME/CFS), and the risk of severe acute respiratory syndrome coronavirus 2 (SARS-CoV-2) Omicron BA.5 subvariant in the context of highly vaccinated populations.

\bsni
% ----------------------------------------------------------------------
% ME/CFS
ME/CFS is a complex disease of unknown cause. The lack of biomarkers for the unequivocal identification of ME/CFS leaves its diagnosis to be made under a degree of uncertainty, relying solely on the assessment of symptoms and the exclusion of other possible diseases that could explain fatigue. Some symptoms, such as long-lasting and profound fatigue that is not alleviated by rest, post-exertional malaise, unrefreshing sleep, or multi-joint pain, are generally required for the diagnosis. However, the spectrum of required symptoms can vary depending on the case definition used, and there are currently more than 20 ME/CFS case definitions proposed.
% This results in a heterogeneous group of diagnosed patients.

% ----------------------------------------------------------------------
% misdiagnosis
The first aim of this thesis is to investigate the diagnostic agreement among four commonly used ME/CFS case definitions and study the overall reproducibility of case-control association studies under the assumption that misdiagnosed individuals may be included in the cohort of patients. I simulated data from different scenarios as a function of sample sizes and the strength of the association between candidate risk factor and the disease. The influence of sensitivity and specificity in the outcome of usual serological tests was also taken into account, with possible misclassification events for false negative or false positive tests. In Chapter~\ref{chapter:statical-challenges-2021}, results showed that although most patients are diagnosed by all, there is no complete agreement between case definitions, thus reinforcing the hypothesis of other misdiagnosed illnesses being included in research studies. In Chapter~\ref{chapter:misdiagnosis-serology-2022} and Chapter~\ref{chapter:impact-misdignosis-2023}, the simulated studies under the assumption of misdiagnosis suggested that the current studies on ME/CFS research have suboptimal statistical power to detect potential true associations with candidate causal factors consistently. Some improvements are suggested, such as increasing sample sizes, reporting the case definitions and exclusion criteria implemented, and focusing on subgroups of more similar ME/CFS patients with a specific pathological mechanism.
% \red{tenho de dizer qual é a recomendação}


% ----------------------------------------------------------------------
% serology
There is a growing body of epidemiological studies where patients report an acute infection as the origin of the disease.
In accordance, studies have found evidence of associations with certain infections, suggesting ME/CFS as a disease with immune dysregulation, similar to autoimmune conditions such as multiple sclerosis (MS), rheumatoid arthritis or, more recently, Long Covid.

% ----------------------------------------------------------------------
% stratification 1
The second aim of this thesis is to explore the relationship between ME/CFS and past viral infections by herpesviruses, stratifying patients by infection trigger or symptom severity. In Chapter~\ref{chapter:2022-revisiting-igg}, I analysed previously published microarray data from IgG antibody responses to Epstein-Barr virus (EBV) antigens in ME/CFS patients and healthy controls and identified two candidate antigens (EBV proteins EBNA4\_0529 and EBNA6\_0070) inducing increased antibody production in patients with an infection trigger. Logistic regression with an interaction between the specific antibodies and age/gender displayed the ability to classify the subgroup of individuals reporting an infection trigger, with high sensitivity and specificity. This indicates that the two EBV antigens could potentially be used as biomarkers for the diagnosis of ME/CFS patients with a putative infection trigger. This finding corroborated the proposal of different subtypes of ME/CFS, reinforcing the strategy of stratifying patients adequately.
Moreover, since EBV-derived antigen EBNA6\_0070 had high sequence homology with human proteins, there is the potential involvement of molecular mimicry in the pathogenesis of the disease.

% ----------------------------------------------------------------------
% stratification 2
In Chapter~\ref{chapter:2023-sym-and-herpesvirus}, analysis of IgG antibody responses against six different herpesviruses in ME/CFS and MS patients revealed distinct antibody-symptom associations between the two conditions. Notably, symptoms from the immunological domain (sore throat, tender glands, and flu-like symptoms) were the main symptoms differentiating the ME/CFS infection trigger subgroups and MS. However, antibody-symptom associations were more heterogeneous when studying the stratified ME/CFS, being generally clearer in the MS control group. Analysing the population of ME/CFS patients showed a link between exposure to herpes simplex virus-1 (HSV1) and experiencing more exacerbated symptoms from the neurocognitive domain. Interestingly, this relation between a neurotropic virus and the severity of symptoms from the neurocognitive domain was also found in Chapter~\ref{chapter:2024-sym-domains}, where ME/CFS patients were grouped by the combined severity of symptoms related to seven specific domains (PEM, immunological, neurocognitive, neuroendocrine, autonomic, neurophysiological, and pain). Once again, these results suggest the possibility of ME/CFS being an umbrella term, encompassing different specific subgroups with similar symptomatology. In this line, the analysis show that stratification of ME/CFS patients enables for a better understanding between different viral infections and the (chronic) activation or dysregulation of particular mechanisms.

% ----------------------------------------------------------------------
% ME/CFS and Covid-19
The project of this thesis was developed during the coronavirus disease 2019 (Covid-19) pandemic. During this period of public health emergency, adjustments were made to accommodate research related to the increased risk of ME/CFS patients towards a SARS-CoV-2 infection, relative to healthy controls. A parallel aim of the thesis is then to study whether the expression of the human angiotensin-converting enzyme 2 (ACE2), the major cell entry receptor for SARS-CoV-2, is altered in patients. In Chapter~\ref{chapter:2021-ace-ace2}, I performed a meta-analysis of public data on CpG DNA methylation and gene expression of this enzyme and its homologous ACE protein in peripheral blood mononuclear cells. The results revealed decreased methylation levels of four CpG probes in the \textit{ACE} locus and one single CpG probe in the promoter region of the \textit{ACE2} gene, suggesting increased expression of the respective genes. Conversely, the meta-analysis revealed decreased expression of \textit{ACE2} but not \textit{ACE} in patients when compared to healthy controls. The results were not particularly clear to provide a definitive answer. However, the finding of increased \textit{ACE:ACE2} ratio in patients was concerning, as it can promote vasoconstriction that could lead to increased production of reactive oxygen species and inflammation.

\bsni
% ----------------------------------------------------------------------
% Covid-19 subvariants intro
Nearing the end of 2021, the rise in cases from the SARS-CoV-2 Omicron variant led to concerns about the protection conferred by vaccines and boosters being used, as they were adapted from early lineages of the virus. Omicron displayed an enhanced ability to evade immunity from previous variants. After a period of dominance from Omicron BA.1 and BA.2 subvariants, during the second half of 2022 Portugal became one of the first countries with Omicron BA.5 as the dominant variant. At this time, new adapted vaccines under development were based on BA.1, meaning that there was a need to study if vaccines and previous Omicron infections would grant effective protection against infections (and reinfections) from the new Omicron subvariant.
% ----------------------------------------------------------------------
% covid analysis
The last aim of this thesis is to study the protection effectiveness and stability over time from infections with past variants and subvariants, towards the Omicron BA.5, using the Portuguese population as a case study of a highly vaccinated population---with a vaccination cover over 98\% of individuals 12 years and older by the end of 2021. In Chapter~\ref{chapter:2022-covid19-01}, results showed that hybrid immunity (vaccination + single past infection) reduced the risk of BA.5 overall and previous infection with subvariants BA.1/BA.2 conferred the highest protection efficacy overall (protection effectiveness by different single infection variants in relation to the uninfected group: Wuhan-Hu-1 51.6\%, Alpha 54.8\%, Delta 61.3\%, BA.1/BA.2 75.3\%). In addition, in Chapter~\ref{chapter:2023-covid19-02}, I estimated that this additional immunity wanes over time, with the relative risk towards a BA.5 infection rapidly increasing from approximately 0.06 to 0.35 between three to eight months following a BA.1/BA.2 infection, only to stabilise around similar values (relative risk approximately 0.37) after that period and to eight months post-infection. These findings suggested that in a population with very high vaccine coverage, BA.1 adapted vaccine boosts would be successful in reducing the risk of breakthrough  BA.5 infections.

\bsni
% ----------------------------------------------------------------------
% Concluding remarks
A large number of patients infected with SARS-CoV-2 has developed a post-acute infection syndrome, Long Covid. Notably, the symptoms that characterise this disease overlap with those of ME/CFS, and it has been discussed whether there is an etiological link between the two. However, there is a clear association between an infection and Long Covid. Still, these similarities reinforce the need to understand better the relationship and impact of external agents, such as viruses, on the possible causes of homeostatic imbalances that could lead to ME/CFS and similar diseases.

% (2) specific background
% (3) knowledge gap
% (4) "in this thesis I have shown..."
% (5) results with key, concrete values
% (6) meaning of results

\vfill

% \noindent\textbf{\Large Keywords:} k1, k2, ...
\noindent{\usefont{T1}{qhv}{b}{n}\selectfont\textbf{\large Keywords:}} Myalgic encephalomyelitis/Chronic fatigue syndrome; Misdiagnosis; Patient stratification; SARS-CoV-2 Omicron; Hybrid immunity
