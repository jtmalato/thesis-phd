\section*{Resumo}
\addcontentsline{toc}{section}{Resumo}
\mark{{Resumo}}

% \chaptermark{Resumo}
% \sectionmark{Resumo}
%%%%%%%%%%%%%%%%%%%%%%%%%%%%%%%%%%%%%%%%%%%%%%%%%%%%%%%%%%%%%%%%%%%%%%%%
% Abstract text, 1200--1500 words minimum. Maximum 5 keywords.
% ----------------------------------------------------------------------
% Intro

A presente tese descreve um trabalho de investigação dividido em dois tópicos. No primeiro tópico estudaram-se estratégias de investigação e estratificação de doentes com encefalomielite miálgica/síndrome de fadiga crónica (EM/SFC). O segundo tópico estuda o risco da subvariante Ómicron BA.5 do coronavírus da síndrome respiratória aguda grave 2 (SARS-CoV-2) em contexto de populações com elevada cobertura vacinal.

\bsni
% ----------------------------------------------------------------------
% ME/CFS
A EM/SFC é uma doença complexa e de etiologia e patofisiologia desconhecidas. A falta de marcadores biológicos para a identificação inequívoca da EM/SFC torna o seu diagnóstico incerto, baseando-se unicamente na avaliação de sintomas e na exclusão de outras doenças que possam justificar o cansaço crónico.
Alguns sintomas são recorrentemente observados e necessários para o diagnóstico da doença. Exemplos incluem a fadiga profunda prolongada e não aliviada pelo repouso, o mal-estar pós-esforço (do inglês \textit{post-exertional malaise}, sigla PEM), o sono não reparador, ou as dores musculares ou poliarticulares. No entanto, o espectro de sintomas avaliados pode variar dependendo dos critérios de diagnóstico implementados. E atualmente existem mais de 20 definições propostas para a EM/SFC.
% \red{Inevitavelmente, esta variabilidade de critérios resulta num conjunto heterogéneo de doentes diagnosticados.}

% ----------------------------------------------------------------------
% misdiagnosis
O primeiro objetivo desta tese passa por investigar a concordância entre quatro dos critérios de diagnóstico mais utilizados em EM/SFC e estudar a reprodutibilidade de estudos de associação do tipo caso-controlo, sob a suposição de que indivíduos com falso diagnóstico (do inglês \textit{misdiagnosis}) podem ser incluídos na coorte de doentes. Para tal, simulei dados relativos a vários cenários em função de diferentes tamanhos amostrais ou diferentes forças de associação entre possíveis fatores de risco e a doença. Também considerei a influência de parâmetros como a sensibilidade e especificidade dos testes serológicos usuais, uma vez que estudos serológicos também podem gerar falsos negativos ou falsos positivos. No Capítulo~\ref{chapter:statical-challenges-2021}, os resultados mostraram que embora a maioria da população suspeita de EM/SFC seja diagnosticada pelos critérios de diagnóstico considerados, não existe uma concordância total entre eles, o que reforça a hipótese de influência por falso diagnóstico de doentes. No Capítulo~\ref{chapter:misdiagnosis-serology-2022} e Capítulo~\ref{chapter:impact-misdignosis-2023}, os estudos de simulação realizados assumindo o pressuposto de falso diagnóstico sugerem que os estudos atuais em EM/SFC têm um poder estatístico insuficiente para detetar, de forma consistente, potenciais associações a fatores de risco de interesse. A fim de melhorar a reprodutibilidade dos estudos nesta doença, algumas recomendações são sugeridas, tal como o aumento do tamanho das amostras de ambos os coortes, a comunicação e reporte dos critérios de diagnóstico e de exclusão aplicados, e o foco em subgrupos de doentes com EM/SFC mais semelhantes entre si e com um eventual mecanismo patológico específico.

% ----------------------------------------------------------------------
% serology
Existe um número crescente de estudos epidemiológicos em que os pacientes reportam uma infeção aguda como origem da doença. Em concordância, investigações têm encontrado possíveis relações que sugerem uma associação entre a origem da doença e infeções. Isto sugere que a EM/SFC poderá ser uma doença com desregulação do sistema imune, aproximando-a de doenças autoimunes como a esclerose múltipla (do inglês \textit{multiple sclerosis}, sigla MS) ou a artrite reumatoide.

% ----------------------------------------------------------------------
% stratification 1
O segundo objetivo desta tese é explorar a relação entre EM/SFC e infeções virais anteriores por herpesvírus, estratificando os doentes de acordo com o reporte de infeções como causa da doença ou pela severidade dos sintomas apresentados. No Capítulo~\ref{chapter:2022-revisiting-igg} analisei dados de \textit{microarray} previamente publicados relativos a respostas de anticorpos IgG aos antigénios do vírus Epstein-Barr (VEB) em doentes com EM/SFC, comparando-os com controlos saudáveis. Identifiquei dois antigénios candidatos (proteínas do VEB EBNA4\_0529 e EBNA6\_0070) que induziam maior produção de anticorpos em doentes EM/SFC com infeção na origem da doença. O uso de uma regressão logística com uma interação entre estes anticorpos e idade/sexo demonstrou boa capacidade de classificação para o subgrupo de doentes com origem numa infeção, com sensibilidade e especificidade elevadas. Estes resultados sugerem que os dois antigénios do VEB podem ser utilizados como biomarcadores para o diagnóstico da EM/SFC em indivíduos com uma infeção anterior como origem da doença. Os resultados corroboram a noção de que a EM/SFC é composta por diferentes subtipos, reforçando a proposta de uma estratificação adequada de doentes. Adicionalmente, uma vez que o antigénio viral EBNA6\_0070 possui elevada homologia com proteínas humanas, pode existir o envolvimento de mimetismo molecular na patogénese da doença.

% ----------------------------------------------------------------------
% stratification 2
No Capítulo~\ref{chapter:2023-sym-and-herpesvirus}, a análise a respostas de anticorpos IgG contra seis herpesvírus diferentes em doentes EM/SFC e MS revelou associações distintas entre anticorpos e sintomas nas duas doenças. Nomeadamente, os sintomas do domínio imunológico (\textit{e.g.}, odinofagia, adenopatia, e síndrome gripal) foram os principais sintomas que diferenciaram os subgrupos EM/SFC com infeção prévia e MS. No entanto, associações anticorpos-sintomas foram mais heterogéneas quando se estudou a EM/SFC estratificada, sendo geralmente mais claras nos grupos de controlo da MS. A análise da população de doentes com EM/SFC mostrou uma ligação entre a exposição ao vírus herpes simplex-1 (HSV1) e a ocorrência de sintomas mais exacerbados do domínio neurocognitivo. É interessante notar que esta relação entre um vírus neurotrópico e a severidade dos sintomas do domínio neurocognitivo foi também encontrada no Capítulo~\ref{chapter:2024-sym-domains}, onde estratifiquei os doentes com EM/SFC de acordo com o perfil de severidade de sintomas, agrupados por domínios específicos (domínios relativos a PEM, imunológico, neurocognitivo, neuroendócrino, autonómico, neurofisiológico e dor). Mais uma vez, estes resultados demonstram a possibilidade de EM/SFC ser uma terminologia que engloba diferentes subgrupos específicos mas com sintomatologia semelhante. Desta forma, esta análise demonstra que estratificação de doentes EM/SFC possibilita uma melhor compreensão da ligação entre diferentes infeções virais e a ativação (crónica) ou desregulação de mecanismos de particulares.

% ----------------------------------------------------------------------
% ME/CFS and Covid-19
Esta tese foi desenvolvida durante a pandemia da doença por coronavírus 2019 (Covid-19). Durante este período de emergência de saúde pública, foram realizados alguns ajustes para acomodar a investigação relacionada com o risco acrescido de infeção por SARS-CoV-2 em doentes com EM/SFC relativamente a indivíduos saudáveis. Assim, um objetivo paralelo da tese é estudar se a expressão da enzima conversora de angiotensina humana 2 (ACE2)---o principal recetor de entrada celular do SARS-CoV-2---está alterada nos doentes. No Capítulo~\ref{chapter:2021-ace-ace2} efetuei uma meta-análise de dados públicos de metilação de sítios CpG no ADN e de expressão genética desta enzima e da sua proteína homóloga, a ACE, em células mononucleares do sangue periférico. Os resultados mostraram a diminuição dos níveis de metilação em quatro sondas CpG no locus \textit{ACE} e numa única CpG na região promotora do gene \textit{ACE2}, sugerindo um aumento da expressão dos respetivos genes. No entanto, a mesma análise demonstrou também que há uma diminuição da expressão da \textit{ACE2} mas não da \textit{ACE} nos doentes, quando comparados com controlos saudáveis. Estes resultados não foram particularmente claros para fornecer uma resposta definitiva. Contudo, a descoberta do aumento na razão \textit{ACE:ACE2} em doentes levantou algumas preocupações, uma vez que esta relação pode promover a vasoconstrição e pode levar ao aumento da produção de espécies reativas de oxigénio e inflamação.

\bsni
% ----------------------------------------------------------------------
% Covid-19
Perto do final de 2021, o aumento dos casos da variante Ómicon do SARS-CoV-2 suscitou preocupações quando à proteção conferida pelas vacinas e reforços utilizados, uma vez que eram adaptadas às primeiras linhagens do vírus.
% Esta imunidade conferida pelas vacinas revelou-se insuficiente para evidar infeções com esta nova variante.
A Ómicron demonstrou uma maior capacidade de se evadir e escapar à imunidade imunidade conferida pelas variantes anteriores. Após um período de dominância das subvariantes da Ómicron, BA.1 e BA.2, durante a segunda metade de 2022 Portugal tornou-se num dos primeiros países sob dominância da subestirpe BA.5. Nesta altura, as novas vacinas em desenvolvimento baseavam-se na BA.1, sendo necessário estudar se as vacinas administradas, juntamente com infeções anteriores por Ómicron, confeririam uma proteção eficaz contra as infeções (e reinfeções) pela nova subvariante da Ómicron.

O último objetivo desta tese é, assim, estudar a eficácia e a estabilidade da proteção ao longo do tempo por infeções anteriores, contra a Ómicron BA.5. Utilizou-se a população portuguesa como caso de estudo de uma população fortemente vacinada---mais de 98\% dos indivíduos a partir dos 12 anos receberam  pelo menos uma primeira vacina até ao final de 2021. No Capítulo~\ref{chapter:2022-covid19-01}, os resultados mostraram que a imunidade híbrida (vacinação + infeção única) reduziu o risco da BA.5 e a infeção por BA.1/BA.2 conferiu a maior eficácia de proteção (eficácia de proteção por diferentes variantes de infeção única em relação ao grupo não infetado: Wuhan-Hu-1 51,6\%, Alpha 54,8\%, Delta 61,3\%, BA.1/BA.2 75,3\%). Posteriormente, no Capítulo~\ref{chapter:2023-covid19-02} estimei a perda de imunidade ao longo do tempo, com o aumento rápido do risco relativo de uma infeção por BA.5 de aproximadamente 0,06 para 0,35 entre três a oito meses após uma infeção por BA.1/BA.2, tendendo a estabilizar após esse período. Estes resultados sugerem que numa população com uma cobertura vacinal muito elevada, os reforços de vacinas adaptadas à BA.1 seriam bem-sucedidos na redução do risco de infeções por BA.5, com uma duração de até oito meses.


% ----------------------------------------------------------------------
% Conclusão
Um número elevado de doentes infetados pelo SARS-CoV-2 desenvolve um síndrome de infeção pós-aguda denominado Covid Longo. Curiosamente, os sintomas que caracterizam esta doença sobrepõe-se aos da EM/SFC e discute-se até se não será apenas um novo fator etiológico para a doença, sendo que para o Covid Longo existe uma clara ligação a uma infeção. Estas semelhanças reforçam a necessidade de compreender melhor a relação e impacto de agentes externos, nomeadamente vírus, nas possíveis causas de desequilíbrios homeostáticos que podem conduzir a EM/SFC ou condições semelhantes.
\vfill

\noindent{\usefont{T1}{qhv}{b}{n}\selectfont\textbf{\large Palavras-chave:}} 
Encefalomielite miálgica/Síndrome de fadiga crónica; Falso diagnóstico; Estratificação de pacientes; Ómicron SARS-CoV-2; Imunidade híbrida